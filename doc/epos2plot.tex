\documentclass{article}
\usepackage{graphics,eurosym,latexsym}
\usepackage{listings}
\lstset{columns=fixed,basicstyle=\ttfamily,numbers=left,numberstyle=\tiny,stepnumber=5,breaklines=true}
\usepackage{times}
\usepackage[round]{natbib}
\bibliographystyle{plainnat}
\oddsidemargin=0cm
\evensidemargin=0cm
\newcommand{\be}{\begin{enumerate}}
\newcommand{\ee}{\end{enumerate}}
\newcommand{\bi}{\begin{itemize}}
\newcommand{\ei}{\end{itemize}}
\newcommand{\I}{\item}
\newcommand{\ty}{\texttt}
\newcommand{\kr}{K_{\rm r}}
\textwidth=16cm
\textheight=23cm
\begin{document}
\title{\ty{epos2plot}: Plot \ty{epos} Results}
\author{Bernhard Haubold\\\small Max-Planck-Institute for Evolutionary Biology, Pl\"on, Germany}
\maketitle
\section{Introduction}
\ty{Epos2plot} converts \ty{epos} output to a format
ready for plotting.
\section{Getting Started}
\begin{itemize}
\item Download \ty{epos2plot} or update an existing copy
  \begin{verbatim}
go get -u  github.com/evolbioinf/epos2plot
\end{verbatim}
\item Install \ty{epos2plot}  
\begin{verbatim}
go install github.com/evolbioinf/epos2plot
\end{verbatim}
\end{itemize}
\section{Examples}
The example data is based on Figure 2a of \cite{liu15:exp}: The population size
is 10,000 and 10 sequences of length 10Mbp were simulated with 1000
replications using the coalescent simulator
\ty{mspms} \citep{kel16:eff}. Its output was
converted to site frequency spectra using
\ty{sfs}, and the site frequency
spectra in turn were converted to population sizes using
\ty{epos} (both also available from \ty{github.com/evolbioinf}):
\begin{verbatim}
for i in $(seq 1000); do
	mspms 10 1 -t 12310 -r 9750 10000000 -eN 0.066 0.3 | 
	sfs -f |
	epos -u 1.2e-1
done > fig2a_10.epos
\end{verbatim}
Copy this data to your working directory
\begin{verbatim}
cp $GOPATH/src/github.com/evolbioinf/epos2plot/data/fig2a_10.epos.bz2 .
\end{verbatim}
and uncompress it
\begin{verbatim}
bunzip fig2a_10.epos.bz2
\end{verbatim}
Display the raw data using \ty{pipePlot}
(see \ty{github.com/evolbioinf}):
\begin{verbatim}
epos2plot -r fig2a_10.epos | 
pipePlot
\end{verbatim}
Instead of plotting raw \ty{epos} results, 
\ty{epos2plot} by default computes 2.5\% and 97.5\% quantiles around the median: 
\begin{verbatim}
epos2plot fig2a_10.epos | 
head
#Time	LowerQ	Median	UpperQ	SampleSize
0	9550	10200	18700	987
463	9550	10200	18700	987
482	9550	10200	18700	987
487	9550	10200	18700	987
491	9550	10200	18700	987
502	9550	10200	18700	987
505	9550	10200	18700	987
506	9550	10200	18700	987
510	9540	10200	18700	987
\end{verbatim}
where \ty{Time} contains the time in generations, 
\ty{LowerQ} the lower quantile of the population size, \ty{Median} its
median, and \ty{UpperQ} its upper
quantile. Notice how these values are close to the expected
10,000. The column \ty{SampleSize}, finally, lists the number of data points from which the
quantiles were computed. As one last example, plot the median:
\begin{verbatim}
epos2plot fig2a_10.epos | 
cut -f 1,3 |
pipePlot
\end{verbatim}

\bibliography{/home/haubold/References/references}
\end{document}

